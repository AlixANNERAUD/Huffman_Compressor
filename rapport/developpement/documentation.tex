\section{Documentation}

Pour la documentation, nous avons utilisé \texttt{Doxygen} qui est un outil de documentation de code source à partir de commentaires apposé dans le code.
Ainsi, pour la documentation utilisateur, les commentaires de documentation ont été écrit dans les fichiers d'entête (\texttt{.h}). 
Quant au commentaires sur les détails d'implémentation, ils ont été écrit dans les fichiers source (\texttt{.c}).
\texttt{Doxygen} permet de générer de la documentation dans plusieurs formats.
Afin de pouvoir intégrer la documentation dans le rapport, nous avons choisi le format \LaTeX.