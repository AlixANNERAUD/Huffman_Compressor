\section{Structure du projet}

\subsection{Arborescence}

Le projet est organisé en plusieurs dossiers et sous-dossiers :

\begin{itemize}
    \item \texttt{rapport} : contient le rapport du projet.
    \begin{itemize}
        \item \texttt{analyse} : contient la partie analyse du rapport.
        \item \texttt{conceptionDetaillee} : contient la partie conception détaillée du rapport.
        \item \texttt{conceptionPreliminaire} : contient la partie conception préliminaire du rapport.
        \item \texttt{developpement} : contient la partie développement du rapport (cette partie).
        \item \texttt{conclusion} : contient la conclusion du rapport.
    \end{itemize}
    \item \texttt{programme} : contient le code source du programme.
    \begin{itemize}
        \item \texttt{bin} : contient l'exécutable et fichiers objets issus de la compilation.
        \item \texttt{doc} : contient la configuration de \texttt{Doxygen} ainsi que la documentation générée.
        \item \texttt{include} : contient les fichiers d'entête (\texttt{.h}).
        \item \texttt{lib} : contient la bibliothèque principale (TAD).
        \begin{itemize}
            \item \texttt{include} : contient les fichiers d'en-tête (\texttt{.h}) de la bibliothèque.
            \item \texttt{src} : contient les fichiers source (\texttt{.c}) de la bibliothèque. 
        \end{itemize}
        \item \texttt{src} : contient les fichiers source (\texttt{.c}).
        \item \texttt{test} : contient les fichiers de test.
        \begin{itemize}
            \item \texttt{include} : contient les fichiers d'en-tête (\texttt{.h}) des tests.
            \item \texttt{src} : contient les fichiers source (\texttt{.c}) des tests.
        \end{itemize}
    \end{itemize}
    \item 
\end{itemize}

\subsection{\texttt{Makefile}}

Le projet étant assez conséquent, il est nécessaire d'automatiser la compilation.
Pour cela, nous avons utilisé \texttt{Make}.
Ce dernier vient exécuter le fichier \texttt{makefile} situé à la racine du dossier \texttt{programme}.
Les commandes disponibles sont les suivantes :
\begin{itemize}
    \item \texttt{make} : exécute la commande \texttt{make run}.
    \item \texttt{make run} : compile et exécute le programme principal.
    \item \texttt{make test} : compile et exécute les tests.
    \item \texttt{make doc} : génère la documentation \texttt{Doxygen}.
    \item \texttt{make clean} : supprime les fichiers objets et l'exécutable.
\end{itemize}