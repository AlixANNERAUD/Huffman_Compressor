\section{Compresser}

\begin{algorithme}
    % Déclaration
    \procedure{compresser}{\paramEntree{source : \binaryFile}, \paramEntreeSortie{destination : \binaryFile}, \paramSortie{error : \booleen}}{}
    % Déclaration
    {stat : \statistics,
        tree : \huffmanTree,
        table : \codingTable}
    % Corps
    {
        \affecter{error}{Faux}
        \instruction{open(source, read)}
        \affecter{stat}{computeStatistics(source)}
        \commentaire{Calcul des statistiques}
        \affecter{tree}{buildHuffmanTree(stat)}
        \commentaire{Construction de l'arbre de Huffman à partir des statistiques}
        \affecter{table}{buildCodingTable(tree)}
        \commentaire{Construction de la table de codage à partir de l'arbre de Huffman}
        \affecter{destination}{binaryFile("compressFile")}
        \instruction{open(destination, write)}
        \instruction{writeHeader(destination, stat)}
        \commentaire{Écriture de l'en-tête (identifiant, statistiques ...) dans le fichier de destination}
        \instruction{compressSourceBytes(source, destination, table)}
        \instruction{close(source)}
        \instruction{close(destination)}
        \affecter{error}{Vrai}
    }
    \espace
    \fonction{computeStatistics}{source : \binaryFile}{\statistics}{}
    {byte : \byte, stat : \statistics}
    {
        \affecter{stat}{statistics()}
        \commentaire{Initialisation des structures de données}
        \instruction{seek(binaryFile, 0)}
        \tantque{non(isEnd(source))}
        {
            \affecter{byte}{readByte(source)}
            \instruction{increaseCount(stat, byte)}
        }
        \retourner{stat}
        }
    \espace
    \fonction{buildHuffmanTree}{stat : \statistics}{\huffmanTree}{}
    {queue : \priorityQueue,
        tree, leaf, rightLeaf, leftLeaf : \huffmanTree,
        byte : \byte,
        count : \naturel}
    {
        \affecter{tree}{huffmanTree()}
        \affecter{queue}{priorityQueue()}
        \pour{i}{0}{255}{}
        {
            \affecter{byte}{i}
            \commentaire{On crée une feuille d'arbre d'Huffman pour chaque octet disctinct présent dans le fichier}
            \affecter{count}{getCount(stat, byte)}
            \sialors{count > 0}
            {
                \affecter{leaf}{newLeaf(byte, count)}
                \instruction{push(queue, leaf)}
            }
        }

        \tantque{non(isEmpty(queue))}
        {
            \affecter{leftLeaf}{pop(queue)}
            \sialorssinon{non(isEmpty(queue))}
            {
                \affecter{rightLeaf}{pop(queue)}
                \commentaire{Si la file contient au moins deux éléments, on les fusionne pour créer un nouvel arbre}
                \affecter{tree}{newTree(leftLeaf, rightLeaf)}
                \instruction{push(queue, tree)}
            }
            {
                \affecter{tree}{leftLeaf}
                \commentaire{S'il ne reste qu'un seul élément dans la file, c'est l'arbre de Huffman}
            }
        }

        \retourner{tree}
    }
    \espace
    \fonction{buildCodingTable}{tree : \huffmanTree}{\codingTable}{}
    {table : \codingTable,
        binaryCode : \binaryCode}
    {
        \affecter{table}{codingTable()}
        \affecter{binaryCode}{binaryCode()}
        \instruction{buildCodingTableRecursive(tree, table, binaryCode)}
        \retourner{table}
    }
    \espace
    \procedure{buildCodingTableRecursive}{\paramEntree{tree : \huffmanTree}, \paramEntreeSortie{table : \codingTable}, \paramEntreeSortie{current\_binary\_code : \binaryCode}}
    {}
    {}
    {
        \sialorssinon{isLeaf(tree)}
        {
            \instruction{add(table, getValue(tree), binaryCode)}
        }
        {
            \instruction{addBit(current\_binary\_code, 0)}
            \instruction{buildCodingTableRecursive(getLeftLeaf(tree), table, current\_binary\_code)}
            \instruction{removeLastBit(current\_binary\_code)}
            \instruction{addBit(current\_binary\_code, 1)}
            \instruction{buildCodingTableRecursive(getRightLeaf(tree), table, current\_binary\_code)}
            \instruction{removeLastBit(current\_binary\_code)}
        }
    }
    \espace
    \procedure{writeHeader}{\paramEntree{destination : \binaryFile}, \paramEntree{stat : \statistics, size : \naturel}}
    {}
    {}
    {
        \instruction{writeString(destination, "HUFF")}
        \commentaire{Écriture de l'identifiant utilisé pour reconnaître les fichiers compressés d'Huffman}
        \instruction{writeNatural(destination, size)}
        \commentaire{Écriture de la taille des données compressées}
        \instruction{writeNatural(destination, getSize(stat) + sizeof(size) + sizeof("HUFF") - 1)}
        \commentaire{Écriture de la taille de l'en-tête (décallage où commencent les données)}
        \instruction{write}
    }
    \espace
    \procedure{compressSourceBytes}{\paramEntree{source : \binaryFile}, \paramEntreeSortie{destination : \binaryFile}, \paramEntree{table : \codingTable}}
    {}
    {binaryCode : \binaryCode, byte : \byte}
    {
    
        
        \instruction{seek(source, 0)}
        \tantque{non(isEnd(source))}
        {
            \affecter{byte}{readByte(source)}
            \affecter{binaryCode}{getCode(table,byte)}
            \instruction{write(destination,binaryCode)} % à changer
        }
        
    }
    \espace
\end{algorithme}
