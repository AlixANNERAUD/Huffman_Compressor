\subsection{Octet}

Un octet est une suite de 8 bits. Ici, il nous servira a manipuler plus facilement les données bit à bit.

\newcommand{\byte}{\textbf{Byte}}

\begin{tad}
    \tadNom{\byte}
    \tadDependances{\bit, \naturel}
    \begin{tadOperations}{byte\_to\_natural}
        \tadOperationAvecPreconditions{byte}{\naturel}{\byte}
        \tadOperation{byte\_to\_natural}{\byte}{\naturel}
        \tadOperationAvecPreconditions{set\_bit}{\tadParams{\byte,\naturel,\bit}}{\byte}
        \tadOperationAvecPreconditions{get\_bit}{\tadParams{\byte,\naturel}}{\bit}
    \end{tadOperations}
    \begin{tadSemantiques}{byte\_to\_natural}
        \tadSemantique{byte}{Crée un octet a partir d'un naturel.}
        \tadSemantique{byte\_to\_natural}{Convertit un octet en naturel.}
        \tadSemantique{set\_bit}{Défini le i-ème bit.}
        \tadSemantique{get\_bit}{Obtient le i-ème bit.}
    \end{tadSemantiques}
    \begin{tadAxiomes}
        \tadAxiome{get\_bit(set\_bit(byte, i, bit), i) = bit}
    \end{tadAxiomes}
    \begin{tadPreconditions}{get\_bit(byte, i)}
        \tadPrecondition{byte(natural)}{natural < 256}
		\tadPrecondition{set\_bit(byte, i)}{i $\leq$ 7}
		\tadPrecondition{get\_bit(byte, i)}{i $\leq$ 7}
    \end{tadPreconditions}
\end{tad}
