\section{Code binaire}

\newcommand{\binaryCode}{\textbf{BinaryCode}}

\begin{tad}
    \tadNom{\binaryCode}
    \tadParametres{}
    \tadDependances{\bit, \naturel, \booleen}
    \begin{tadOperations}{getBit}
        \tadOperation{binaryCode}{}{\tadParams{\binaryCode}}
        \tadOperationAvecPreconditions{setBit}{\tadParams{\binaryCode,\naturel, \bit}}{\tadParams{\binaryCode}}
        \tadOperation{isBinaryCode}{\tadParams{\binaryCode}}{\tadParams{\booleen}}
        \tadOperationAvecPreconditions{length}{\tadParams{\binaryCode}}{\tadParams{\naturel}}
        \tadOperationAvecPreconditions{getBit}{\tadParams{\binaryCode, \naturel}}{\tadParams{\bit}}
    \end{tadOperations}
    \begin{tadSemantiques}{getBit}
        \tadSemantique{binaryCode}{Créer un code binaire vide.}
        \tadSemantique{setBit}{Défini le ième bit d'un code binaire.}
        \tadSemantique{isBinaryCode}{Savoir si c'est un code binaire qui désigne une feuille dans l'arbre de Huffman.}
        \tadSemantique{length}{Renvoi la longueur d'un code binaire.}
        \tadSemantique{getBit}{Renvoi le ième bit d'un code binaire.}
    \end{tadSemantiques}
    \begin{tadAxiomes}
        \tadAxiome{getBit(a,i)=getBit(setBit(setBit(a,i),i),i)}
    \end{tadAxiomes}
    \begin{tadPreconditions}{getBit(code, index)}
		\tadPrecondition{setBit(code, index, bit)}{0 $\leqslant$ i $\leqslant$ 7}
        \tadPrecondition{length(code)}{isBinaryCode(code)}
		\tadPrecondition{getBit(code, index)}{0 $\leqslant$ i $\leqslant$ 7 }
    \end{tadPreconditions}
\end{tad}