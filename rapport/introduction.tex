\chapter{Introduction}

\section{Présentation}

Dans le cadre du cours d'algorithmique avancée et programmation en C dispensé par M. Nicolas DELESTRE,
nous avons réalisé un (dé)compresseur d'Huffman.

\section{Organisation}

\subsection{Outils}

Nous avons utilisé les outils suivant pour réaliser ce projet :

\begin{itemize}
    \item \texttt{Git} et \texttt{GitLab} pour la gestion de version et le partage du code source. N'ayant pas eu de cours sur les branches, nous avons travaillé sur une seule et unique branche \texttt{main}.
    \item \texttt{Visual Studio Code} pour l'écriture du code source et de la documentation
    \item \texttt{Doxygen} pour la génération de la documentation
\end{itemize}

\subsection{Répartition des tâches}

L'affectation des tâches s'est faite par l'intermédiaire des tickets ("issues") de GitLab. 
La répartition des tâches s'est faite de manière à ce que chacun puisse travailler sur tous les aspects du projet. 
Ainsi, une personne qui a été responsable de l'analyse d'une partie, a été responsable de la conception d'une autre partie, puis de l'implémentation d'une autre partie, etc.
Nous avons donc la répartition suivante :

% TODO : mettre un tableau avec les tâches et les personnes qui les ont faites