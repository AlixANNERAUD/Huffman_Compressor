\section{Convention}

Afin de garantir une certaines cohérence dans le code, nous avons décidé d'adopter une convention de programmation.
Cette convention est inspirée de celle utilisée par la communauté C.

\subsection{Langage}

\begin{itemize}
    \item Les identifiants dans le code seront écrit en anglais.
    \item Les commentaires seront écrits en français.
    \item Les acronymes sont proscrits.
\end{itemize}

\subsection{Nommage des identifiants}

\begin{itemize}
    \item Les macro et constante (variables constantes, macro constante, enumerations ...) seront écrites en majuscule et séparées par des tirets du bas. Par exemple : \texttt{MAXIMUM\_LENGTH}.
    \item Les identifiants des types seront écrits en casse de pascal. Par exemple : \texttt{BinaryCode}.
    \item Les identifiants des variables et des fonctions sont écrits en \texttt{casse\_de\_serpent}. Par exemple : \texttt{write\_header}.
    \item Dans le cas où l'identifiants dépend d'un type (excepté les types eux-mêmes), le nom de l'identifiant sera préfixé par le nom du type tout en conservant sa casse d'origine.
    Par exemple : \texttt{binary\_code\_length} dans le cas d'une fonction et \texttt{CODING\_TABLE\_MAXIMUM\_SIZE} dans le cas d'une constante.
\end{itemize}

\subsection{Conventions de programmation}

\begin{itemize}
    \item Les variables globales sont interdites. Cela permet d'éviter les effets de bords et de faciliter la compréhension du code.
\end{itemize}