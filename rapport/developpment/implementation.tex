\section{Implementation}

\subsection{Préconditions}

Afin d'éviter tout comportement indéfini, nous avons vérifié les préconditions à l'exécution des fonctions.
Pour cela, nous avons utilisé \texttt{assert.h} de la bibliothèque standard du C. 
Ainsi, si une précondition n'est pas respectée, le programme s'arrête et affiche un message d'erreur.
Cela permet de détecter une bonne partie des erreurs d'implementation lors du développement.

\subsection{Gestion des erreurs}

Afin de gérer les erreurs au temps d'exécution, nous avons fait le choix de ne pas utiliser \texttt{errno.h} de la bibliothèque standard du C.
Cette étant trop limitée, nous avons préféré <a completer>.
A la place, nous avons créer des énumerations dans les fichiers d'entête (\texttt{.h}) afin de pouvoir gérer les erreurs de manière plus précise.
Nous avons également ajouté des fonctions permettant de convertir ces énumerations en chaîne de caractères afin de pouvoir afficher des messages d'erreurs plus explicites.
Cela permet une gestion des erreurs plus précise et plus lisible.
